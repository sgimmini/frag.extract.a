\documentclass{report}
\usepackage{hyperref}
\renewcommand\labelitemi{\textbf}

\begin{document}

\begin{titlepage}
	\begin{center}
	\vspace*{1cm}
	\Huge

	\textbf{Fragment Extraction from StackOverflow}

	\vspace{0.5cm}
	\LARGE

	Beginner Practical Summer 2019

	\vspace{1.5cm}

	\textbf{Florian, Simon, Tobias}

	\vspace{1cm}
	Supervisor:
	\vfill


	\vspace{1.8cm}
	\Large
	IWR?\\
	Ruprecht-Karls University\\
	\today
	\end{center}
    \end{titlepage}

\section*{Chrome Extension Tests}
Precondition for all test: installed Chrome Extension and VSC Extension if not further specified.
\subsection*{Get Question with one Tag in PopUp}
\begin{itemize}
	\item[Precondition] Opened question with one tag
	\item[Test Steps] Open chrome extension by clicking on extension icon, send fragment to VSC, verify tags
		in VSC.
	\item[Expected Result] No Tag in VSC and closed Chrome PopUp
	\item[Expected Exception] No codeblock on website
\end{itemize}
Test succeeded with this
\href{https://stackoverflow.com/questions/6393943/convert-javascript-string-in-dot-notation-into-an-object-reference}{StackOverflow Question}

\subsection*{Get Question with one Tag with Add To Fragment Button}
\begin{itemize}
	\item[Precondition] Opened question with one tag
	\item[Test Steps] Open PopUp Window by clicking on any codeblocks specified button, send
		fragment to VSC, verify tags in VSC.
	\item[Expected Result] No Tag in VSC and closed Chrome PopUp
	\item[Expected Exception] No codeblock on website
\end{itemize}
Test succeeded with this
\href{https://stackoverflow.com/questions/6393943/convert-javascript-string-in-dot-notation-into-an-object-reference}{StackOverflow Question}

\subsection*{Question with no Codeblock in any Answer}
\begin{itemize}
	\item[Precondition] Question with no codeblock or no answer
	\item[Test Steps] Open Chrome Extension by clicking extension icon and send fragment to VSC
	\item[Expected Result] Extension won't send fragment and hovertext will appear
	\item[Expected Exception] None
\end{itemize}
Tested with: \href{https://stackoverflow.com/questions/21446532/jboss-and-intellij-use-jboss-plugin-to-run-and-deploy-or-use-maven}{StackOverflow Question}.
At tested point unanswered for over 5 years. Succeeded. No differianated test neccessary as there is no Add to Fragment Button.

\subsection*{Sending without VSC-Extension}
\begin{itemize}
	\item[Precondition] Chrome extension installed, no VSC-extension
	\item[Test Steps] Open any thread and send fragment
	\item[Expected Result] PopUp closes, fragment "sent" 
	\item[Expected Exception] None
\end{itemize}
Fragments do not send without a designated host. Nothing changes, nothing happens.

\subsection*{Add last codeblock of any page}
\begin{itemize}
	\item[Precondition] Question thread with at least two codeblocks  
	\item[Test Steps] Click Add to Fragment Button of last codeblock and check whether all data is passed
	\item[Expected Result] All data appearing in popping up window
	\item[Expected Exception] None
\end{itemize}
Test succeeded with this \href{https://stackoverflow.com/questions/442404/retrieve-the-position-x-y-of-an-html-element}{StackOverflow Question}. 
Purpose of the test was to check whether there was an out of range (of the Array of all Codeblocks) error. 

\subsection*{Jump to Fragment, not the first}
\begin{itemize}
	\item[Precondition] Question where model does not predict first codeblock
	\item[Test Steps] Open popup, click on jump to fragment button
	\item[Expected Result] screen should be scrolled to desired postition
	\item[Expected Exception] does not work in popup window, where you get by adding fragment manually
\end{itemize}
Test succeeded with this \href{https://stackoverflow.com/questions/221294/how-do-you-get-a-timestamp-in-javascript?rq=1}{StackOverflow Question}.

%Dummy
\subsection*{}
\begin{itemize}
	\item[Precondition]
	\item[Test Steps]
	\item[Expected Result]
	\item[Expected Exception]
\end{itemize}

\section*{Pair Programming}
\Large Aufgabe: Debugging install\_host.sh\\ \normalsize
Programmierer: Simon, Supervisor: Tobias\\
\\
Start: 11:32
\begin{itemize}
	\item HOST\_NAME geaendert und ''.py'' angehangen
	\item testen - funktioniert nicht
	\item googlen ob Schreibweise des Manifests ausschlaggebend ist
		\subitem manifests haben meist dreifach getrennte Schreibweise: w.x.y.z
	\item nmhost manifest name in andere Schreibweise gebracht:
		\subitem aus manifest\_frag\_extract\_host.json - com.frag.extract.json
		\subitem auch in com.frag.extract.json geändert
	\item Kommunikation funktioniert, host arbeitet nicht
	\item neue Namen in uninstall\_host.sh angepasst
	\item in popup.js connectNative aktualisiert
\end{itemize}
Ende: 11:48\\
\\\\
Programmierer: Tobias, Supervisor: Simon\\

Start: 12:01
\begin{itemize}
	\item chrome extension kann auf python host nicht zugreifen
		\subitem manuell Schreibrechte gegeben
		\subitem funktioniert
	\item googlen wie man Schreib- und Leserechte beim Erstellen von Datein gewährleisten kann
	\item beim kopieren des python Skripts Schreibrechte gegeben
	\item Kommunikation nach Installation funktioniert nun ohne manuelle Rechtevergabe oder Kopien

	\item chromium support fehlt:
		\subitem chromium hat anderen Pfad bei Linux als Chrome
		\subitem zweiten Installationsort herausgefunden: /etc/chromium/native-messaging-host
		\subitem diesen Pfad in die install\_host.sh hinzugefuegt
		\subitem ebenfalls in der uninstall\_host.sh
\end{itemize}
Ende: 12:18


\end{document}
