\documentclass{report}

\begin{document}

\Large Aufgabe: Debugging install\_host.sh\\ \normalsize
Programmierer: Simon, Supervisor: Tobias\\
\\
Start: 11:32
\begin{itemize}
	\item HOST\_NAME geaendert und ''.py'' angehangen
	\item testen - funktioniert nicht
	\item googlen ob Schreibweise des Manifests ausschlaggebend ist
		\subitem manifests haben meist dreifach getrennte Schreibweise: w.x.y.z
	\item nmhost manifest name in andere Schreibweise gebracht:
		\subitem aus manifest\_frag\_extract\_host.json - com.frag.extract.json
		\subitem auch in com.frag.extract.json geändert
	\item Kommunikation funktioniert, host arbeitet nicht
	\item neue Namen in uninstall\_host.sh angepasst
	\item in popup.js connectNative aktualisiert
\end{itemize}
Ende: 11:48\\
\\\\
Programmierer: Tobias, Supervisor: Simon\\

Start: 12:01
\begin{itemize}
	\item chrome extension kann auf python host nicht zugreifen
		\subitem manuell Schreibrechte gegeben
		\subitem funktioniert
	\item googlen wie man Schreib- und Leserechte beim Erstellen von Datein gewährleisten kann
	\item beim kopieren des python Skripts Schreibrechte gegeben
	\item Kommunikation nach Installation funktioniert nun ohne manuelle Rechtevergabe oder Kopien

	\item chromium support fehlt:
		\subitem chromium hat anderen Pfad bei Linux als Chrome
		\subitem zweiten Installationsort herausgefunden: /etc/chromium/native-messaging-host
		\subitem diesen Pfad in die install\_host.sh hinzugefuegt
		\subitem ebenfalls in der uninstall\_host.sh
\end{itemize}
Ende: 12:18

\end{document}