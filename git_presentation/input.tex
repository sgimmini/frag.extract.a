Small overview about the functions of git and the most used commands.
\begin{center}
	\underline{\underline{All covered commands start with \texttt{git}}}\\
\end{center}

\section*{Getting Started}


\subsection*{init} 
initialize a repository\\
optional flag: \texttt{-{}-bare} causing to have 
a repository without a working tree where you push from other repositories \\ 

\subsection*{clone 'url'}
calls \texttt{init} but copies an existing repository\\

\subsection*{remote add origin 'url'}
specify a remote repository electively on a website like \texttt{github.com} or \texttt{gitlab.com}\\

\subsection*{config -{}-global user.name "Name"}
sets an username for all local git repositories\\

\subsection*{config -{}-global user.email E-mail}
sets an E-Mail-address for all local git repositories\\

\subsection*{config -{}-list}
shows all set configurations for the called repository \\


\section*{Basic commands}

\subsection*{status}
gives informations about the current working tree, as un-/staged files and branch\\

\subsection*{add 'filename'/'path'}
add a file to index\\
optional: by '\texttt{*}' adding all tracked files\\

\subsection*{commit}
commit current contends of index. You will be asked for a commit-message in which you describe changes\\
common way: \texttt{-m "\ 'commit message' "} - one line commit\\

\subsection*{fetch}
downloads from remote without any merges on local files\\

\subsection*{pull}
download the newest commit from remote and tries to merge conflicts automatically \\

\subsection*{push}
upload the newest commit to remote\\

\subsection*{checkout}
switches branches if appended by 'branchname'\\
or: removes a staged file from commit if appended by 'filename'\\

\subsection*{branch}
creates a new branch pointing on the same commit as the master\\
used for having a independent working tree, i.e. for an experimental build\\

\subsection*{log}
information about the last commits \\
optional: \texttt{-{}-stat} gives detailed information of commits as changed lines per file \\

\subsection*{merge}
updates a branch to the state of an different one, by default with recursive auto-merging\\


\section*{Advanced/Specific commands}
Some commands you probably only need in a very distinct situation\\
\subsection*{update-index –assume-unchanged 'filename'}
in case to untrack a file which already has been added to the git repository\\

\subsection*{SSH management}
it is possible to use ssh to log in into your chosen remote website. Create a new ssh key by \texttt{ssh-keygen -t rsa -b 4096 -C "your\_email@example.com"}, copy it to clipboard and paste it in settings $\rightarrow$ SSH and GPG keys.\\

\subsection*{config -{}-global credential.helper cache}
if you're using https to clone your repositories it might be useful to cache your credentials saving you from having to enter them every time

\subsection*{cherry-pick}
apply chosen commits to branch instead of all (like with merge)\\

\subsection*{blame}
used to see which contributor changed which line \\


\section*{Useful knowledge}
Useful features of git and knowledge about common problems
\subsection*{The ".gitignore" \ file}
file to exclude specific file types from being tracked.\\
i.e. "*.log" not to have LaTeX log files in the repository\\

\subsection*{Merge Conflict}
\texttt{<{}<{}<{}<{}<{}<{}< HEAD} - start of conflict\\
your changes\\
\texttt{=======} - separator \\
changes in other branch \\
\texttt{>{}>{}>{}>{}>{}>{}> BRANCH-NAME} - end of conflict\\
To resolve the conflict delete the three added lines and decide with lines to keep, delete the others. After that commit.


